\documentclass{article}

% Spacing
\usepackage{setspace}
\setlength{\parskip}{\baselineskip}%
\onehalfspacing{}

% Bibliography
\usepackage[american]{babel}
\usepackage[style=apa, citestyle=apa, backend=biber]{biblatex}
\DeclareLanguageMapping{american}{american-apa}
\addbibresource{references.bib}
\usepackage[babel,threshold=2]{csquotes}

% Title
\title{CMSC 125 Final Project}
\date{June 2, 2020}
\author{Tumulak, Patricia Lexa U.\\Valles, Oscar Vian L.}

% Table of Content
\setcounter{tocdepth}{3}

% Colors
\usepackage[dvipsnames]{xcolor}

% Links
\usepackage{hyperref}
\hypersetup{
	pdftitle={CMSC 125 Final Project},
	pdfstartview={FitH},
	colorlinks=true,
	linkcolor=black,
	citecolor=black,
	filecolor=black,
	urlcolor=MidnightBlue,
}


\begin{document}

\pagenumbering{gobble}
\maketitle
\newpage
\tableofcontents
\newpage
\pagenumbering{arabic}

\section{Fedora}
Fedora is a Linux distribution that is built by the Fedora Project and is
sponsored by Red Hat. It advertises itself as an “innovative, free, and open
source platform for hardware, clouds, and containers”. The Fedora Project offers
multiple editions of their operating system based on different use cases. Fedora
Workstation is one of these editions. It is used as operating systems for laptops
and desktop computers. Fedora Server, as its name suggests, is a carefully
crafted version that is tailored for server use. Lastly, Fedora IoT is an
Internet of Things (IoT) focused operating system that serves as a backbone for
IoT ecosystems~\parencite{fedora}.

\subsection{Author(s)}
Fedora is built by the Fedora Project and is sponsored by Red Hat. The Fedora
Project describes themselves as “a community of people working together to build
a free and open source software platform and to collaborate on and share
user-focused solutions built on that platform”.

The project is made up of different leadership groups based on different
functional areas that may change over time. Examples of these groups include the
Fedora Council, which is the topmost governing body of the project. It is made
up of members from the entire community that are either appointed or elected.
Another group is the Fedora Engineering Steering Committee which manages and
directs the technical features of the distribution. Various other smaller groups
exist, like the Fedora Mindshare Committee which manages community growth and
support. These smaller groups are categorized under Working Groups, Subprojects,
and Special Interest Groups. \parencite{fedora_leadership}

\subsection{History}
Fedora Linux started out as a Computer Science academic project by Warren Togami
in 2002 at the University of Hawaii. It got its name Fedora, from the hat used
in the logo of Red Hat at the time. It was created to distribute additional
software for RHL from a central repository that was managed by community
volunteers \parencite{fedora_togami}.

Fedora Linux was then merged with RHL when Red Hat announced the open
development process on July 21, 2003. By then, Red Hat had been developing RHL
for nearly 9 years. On November 5, 2003, Red Hat released Fedora Core 1 to the
world. Since then, Red Hat bases their enterprise Linux solution, Red Hat
Enterprise Linux (RHEL) on release from Fedora.

Since the start of Fedora, two major versions of the operating system have been
released every year. To date, there have been 34 versions released, with the
35th version under development. Some notable versions of Fedora include the
following: Fedora 7 and Fedora 21 \parencite{redhat_history}.

Fedora 7 was released on May 31 2007. This version moved from calling releases
of Fedora as Fedora Core. It was shortened to Fedora. This version also marked
the first time all of its development was 100\% by the community. This was made
possible by merging all of the code into a single external repository. This was
also the first Fedora release that allowed distribution with Live CD/DVD, making
it easier to share and install the operating system \parencite{fedora_7}

Fedora 21 was released on December 9, 2014. This was the version where Fedora
started creating new flavors of Fedora based on different use cases. These
flavors were the following: Cloud, Server, and Workstation. Cloud is the flavor
to use in private cloud environments. Server is a flavor used for a wide array
of application stacks. It can be used as a web server, file server, and database
server, among others. Workstation is optimized for use on desktops and laptops.
It includes a built desktop environment, KDE, for ease of use
\parencite{fedora_21}. Other releases include general improvements which includes
bumping version numbers from upstream packages, updating the Linux kernel to the
latest stable version, and adding support for additional architectures.


\subsection{Influence}
Fedora has greatly influenced other distributions. The following active
distributions were released as remixes or modifications of Fedora: Berry Linux,
FX64 Linux, MontanaLinux, and Network Security Toolkit. Other remixes have also
been created but they have not been updated for a long time. These outdated
remixes include: Hanthana, FedBerry, Amahi, and Korora. \parencite{fedora_remix}

In addition, Fedora has a great influence on the Linux community as a whole.
Fedora has used a lot of software first which was then parroted by the Linux
community. These include PulseAudio and SELinux. Fedora has also created
NetworkManager and D-bus. Now, NetworkManager is the default network
configuration tool for various Linux distributions. D-bus is used by GNOME and
KDE, the two most commonly used desktop environments today.

\subsection{Motivation}
The initial release of Fedora Linux was motivated to fix the limited number of
packages available in RHL. Since the merge, Fedora has a set of values that
guide them throughout the development process. They have named these values the
“Four Foundations”: Freedom, Friends, Features, First.

Freedom means that the Fedora Project is dedicated to free and open source
software and content. This means choosing free alternatives to closed source
software and to reduce the number of proprietary code on the project.

Friends means that the project cultivates a strong and caring community that
allows anyone to help, as long as these people believe in the same core values
that the project believes in. Technical skill is not a prerequisite.

Features means that the project cares about creating and developing excellent
software. Fedora aims to create flexible and powerful software that empowers
users.

First means that the project is committed to innovation. The project believes in
creating better solutions to the problem first, even if that means long-term
stability is compromised. \parencite{fedora_mission}

\subsection{Objective and Purpose}
The Fedora Council creates short, medium, and long term objectives of the
organization. The long term objective of Fedora, according to their mission, is
“to create an innovative platform for hardware, clouds, and containers that
enables software developers and community members to build tailored solutions
for their users.” \parencite{fedora_mission}

The current medium and short term objective of the project is to revamp the
community outreach teams within Fedora. This goal involves identifying
struggling teams and bringing them together under CommOps in a clear structure
to better serve the community.

Past medium and short term objectives that the project were able to accomplish
include increasing involvement in the educational sector, modularization of the
operating system, a flavor specifically for Internet of Things, and
minimization. \parencite{fedora_objective}

\section{openSUSE}
openSUSE is a Linux-based operating system and distribution sponsored by SUSE.
It is developed by the openSUSE project which is a global community that
collaborates on the development of openSUSE (formerly called SUSE Linux), aims
to ``promote Linux everywhere'', and provide and promote free and open source
software. The project itself is controlled by the community --- almost nothing
happens within the operating system without the express permission of the end
user. openSUSE also allows users to select packages to include in their flavor
of openSUSE which is unique among major Linux distributions. It is entirely free
and open source with users able to choose between two major offerings --- openSUSE
Leap and openSUSE Tumbleweed. Leap is the stable version with releases almost
every year and is intended for users who are relatively new to Linux or those
who want a usable and stable OS with regular updates. Tumbleweed is the rolling
release that provides the latest stable updates of the software regularly
whenever they are available. It is intended for the more advanced Linux users.

\subsection{Author(s)}
openSUSE is a community project --- described as a “do-ocracy” wherein those who
do the work are also the ones who decide what happens. Contributions on the OS
come from the community. SUSE, the project’s primary sponsor, contributes to the
project just like any other contributor.

It is self-organized and has three organizational units: the openSUSE Board
(which consists of 5 members elected every 2 years and the chairman who is
elected by SUSE), Election Officials, and Membership-Officials.

According to Brown, former openSUSE chairman, contributors to the project are
expected to collaborate. If there are differing ideas between contributors, they
are expected to reach a compromise. Otherwise, if a conflict arises, the board
steps in to mediate.

\subsection{History}
The company SUSE was founded on September 2, 1992 in Germany under the name
Gesellschaft für Software- und Systementwicklung mbH (S.u.S.E. GmbH), which
means ``Software and System Development, Inc.''. With their release of S.u.S.E.
Linux 1.0 in 1994, SUSE is one of the oldest existing GNU/Linux distributions
and is also one of the leading distributions today. SUSE was then acquired by
Novell in 2003.

In 2005, Novell launched openSUSE --- a community-based Linux distribution
following in the footsteps of Red Hat Inc.\ and its Fedora Project Linux
distribution. The openSUSE project was launched with the goal of involving the
community more and opening up development.

The first stable release from the openSUSE project was SUSE Linux 10.0 on
October 6, 2005 which was released as a freely downloadable ISO image. SUSE
Linux 10.1 was subsequently released on May 11, 2006. On their third release,
the distribution was renamed from SUSE Linux to openSUSE with openSUSE 10.2. For
10.2, the developers focused on reworking menus that launched KDE and GNOME,
making ext3 the default file system, improving power management framework and
the package management system.

Until version 13.2 released on November 4, 2014, the project released stable
versions with separate maintenance streams from SUSE Linux Enterprise. These
versions included but were not limited to updates in GNOME, KDE, the file system
from ext3 to Ext4, and Linux kernel version as well as improvements to YaST and
other new features.

In 2014, openSUSE Factory, which is the project’s development branch, was
stabilized enough to become a stable and tested rolling release distribution,
called openSUSE Tumbleweed. Tumbleweed is the flagship of openSUSE with new
software each day. It forms the base for SUSE Linux Enterprise Server and
Desktop (SLES and SLED). Since late 2015, openSUSE split into two main offerings
- Tumbleweed and Leap. Leap is the classic stable distribution which uses SUSE
Linux Enterprise for the core system. It also has a form of long term support
with major service packs released every year and major releases only happening
every 3--4 years.

The next latest release is openSUSE Leap 15.3 wherein the repository for Leap
and SLE will be merged, sharing the same source code and the exact same binary
packages.

\subsection{Influence}
openSUSE is considered one of the most complete Linux distributions that is
available today. It consistently is among the top five downloads in
distrowatch.org. openSUSE Leap is particularly popular with Linux developers,
system administrators, and software vendors due to its stability and reliability
as well as regular stable updates. The branches of openSUSE are also the basis
for SUSE Linux Enterprise releases. In 2012, 80\% of Linux mainframe systems ran
SUSE Linux and half of the world’s largest supercomputer clusters ran SUSE
Linux.

\subsection{Motivation}
openSUSE was first started with the purpose of making a community-based Linux
distribution that is both free and open source. This allows individuals to
contribute to the project as testers, writers, translators, developers,
usability experts, among others.

\subsection{Objective and Purpose}
The openSUSE project aims to provide easy access to software that is free and
open source. It also aims to propagate the use of Linux. Free in “Free Software”
refers to freedom and not price. The license on the software created and
promoted permits users to study and modify the software however they want.
Community members are free to pursue new initiatives and ideas - allowing for
innovation.

According to their guiding principles, they aim to “create the best Linux
distribution in the world” and to “foster the success of Linux everywhere” with
the largest community which also provides the primary source for free software.
Another aim is to create a distribution that is stable and easy to use and that
is completely multi-purpose for everybody - for both users and developers, for
desktop and server use, and for beginners and experienced users.

Aside from developing the Linux based distribution, the openSUSE project also
develops tools such as Open Build Service (an open and complete distribution
development platform), openQA (an automated testing tool for operating systems),
Kiwi (an OS image builder for Linux supported hardware platforms), YaST (a Linux
OS setup and configuration tool), and OSEM (an event management app). The
project guiding principles also affirms that while doing all this work, it’s
motto is to “Have a lot of fun”.

\section{Feature Discussion}
\subsection{Kernel}
Fedora and openSUSE both use the Linux Kernel. Fedora does not deviate from the
upstream kernel source. The same is also true for openSUSE.@Therefore, there is
no difference between the kernel of both operating systems, other than version
differences.

The Linux kernel is a monolithic kernel --- it is a large program composed of
several components. It can also dynamically load and unload portions of itself
on demand --- these portions are called modules and are usually device drivers. It
is also a preemptive kernel and it supports multithreaded applications. In
addition, multiprocessor systems are also supported by the linux kernel.
\parencite[3]{linux}

The following sections discuss how the kernel manages the processor, memory,
files, resources, and concurrency.

\subsection{Processor Management}
Both operating systems do not modify how Linux handles processor management,
however, the supported processors differ between the two. Fedora primarily
supports 64-bit processors, with support for other processors handled by the
community \parencite{fedora_arch}. openSUSE supports both 32-bit and 64-bit
processors. \parencite{opensuse_arch}.

In Linux, processes act as entities where resources such as CPU time and memory
are allocated to. Linux uses lightweight processes as a method to support
multithreading. Lightweight processes share resources and whenever one of these
resources modifies a shared resource, the other process is notified of the
changes. These processes are bundled together in a thread group. This thread
group implements a multithreaded application and acts as a single process when
interacting with certain system calls. The state of each process is stored in a
process descriptor. \parencite[79-81]{linux}

Linux has three different mechanisms to create different types of processes with
different ways parent and child processes can access resources. (1) Copy On
Write enables the parent and child to read the same physical pages. Whenever one
process writes to a page, the kernel copies the content into a new page that is
assigned to a writing process. (2) Lightweight processes created through
\texttt{clone()} allows parent and child to share many per-process data. (3)
\texttt{vfork()} creates a process that shares memory address space of the
parent process. The parent process is blocked from execution until the child
exits or executes a new program \parencite[114-115]{linux}

Processes can be destroyed through two ways: \texttt{exit\_group()} and
\texttt{\_exit()}. \texttt{exit\_group()} terminates a full thread group while
\texttt{\_exit()} terminates a single process. The kernel is not allowed to
discard data until the process terminates. In a case where the parent process
terminates before their children, the children become orphans, which are then
adopted by the init process. Whenever the init process terminates one of its
children, the orphans will also be terminated at this point.
\parencite[126-127]{linux}

Scheduling on Linux is based on exactly what kind of process is being scheduled.
There are two types of processes, Real-time and conventional processes
\parencite[262-263]{linux}.

Real-time processes are scheduled using two algorithms, First-In-First-Out and
Round Robin. The scheduler also takes into account the priority of the process.
The process is replaced only if: a higher priority is in the queue, the process
performs a blocking operation, the process is stopped or killed, the process
yields, or it has exhausted its time quantum \parencite[265-266]{linux}.

Conventional processes are scheduled using Completely fair scheduling (CFS). CFS
is different from conventional preemptive scheduling where time slices are taken
and priorities are known at the start. In this type of scheduling, the time
given to a process is computed dynamically. In this scheduler, a target latency
is set, usually 20ms. This target latency is then divided by the entire queue of
runnable processes. In essence, each task is given a 1/N ms slice of the
processor, where N is the number of runnable processes in the queue. However,
this is different from conventional time sharing algorithms because the slice
depends on the number of processes in the queue, making it dynamic. As tasks are
removed from the queue, the remaining processes get a greater slice of time.

In addition a nice value is used to weigh the slice. If a process has
low-priority nice, it is given a fraction of the 1/N slice, and if it has a
greater nice value, it is given a greater fraction of the slice. Do note that
the nice value does not determine the 1/N slice, but it only modifies it per
process.

Preemption in CFS occurs whenever the weighted slice is finished. However there
is a minimum time spent for each process. This minimum time spent is known as
the minimum granularity. This limits the amount of context switching that occurs
as too much context switching incurs a lot of overhead.

This type of scheduling schedules by thread rather than by process. If a
multithreaded application has N threads, then N scheduling actions are taken to
schedule each thread. \parencite{kalin_2019}

\subsection{Memory Management}
Both operating systems do not modify how Linux handles processor management,
however, the supported processors differ between the two. Fedora primarily
supports 64-bit processors, with support for other processors handled by the
community \parencite{fedora_arch}. openSUSE supports both 32-bit and 64-bit
processors. \parencite{opensuse_arch}.

This means that openSUSE, on a 32-bit processor, has more limitations in
accessing and allocating memory. A 32-bit openSUSE install is limited to 4 GB of
addressable space and 3.2 GB of RAM. These limitations do not affect 64-bit
openSUSE installs and all Fedora installs. \parencite{guru99}

Linux uses paging with a 4 KB page frame size as the standard memory allocation
unit, rather than a 4 MB page size that may be used in some processors. This is
because of two reasons: Page Fault exceptions are easily interpreted, and data
transfer between main memory and disk are more efficient when page size is
small. \parencite[294-295]{linux}

In cases where memory access time is not uniform, Linux supports the Non-Uniform
Memory Access (NUMA) model. This type of model is used in some architectures,
like MIPS, which both Fedora and openSUSE supports. \parencite[297-298]{linux}

Linux also manages memory in zones. There are three zones, \texttt{ZONE\_DMA},
\texttt{ZONE\_NORMAL}, and \texttt{ZONE\_HIGHMEM}. These three zones were
created since some architectures limit the way page frames can be utilized. Old
Direct Memory Access (DMA) processors for ISA buses are only able to address the
first 16MB of RAM and 32-bit computers cannot access all physical memory
locations due a limitation in the size of the linear address space
\parencite[299]{linux}. These zones are used by the Zoned Page Frame Allocator.
This component handles requests for allocation and deallocation of dynamic
memory. \parencite[302]{linux}

To reduce external fragmentation, Linux uses the Buddy System Algorithm where
free page frames are grouped into 11 lists of blocks which contain groups of 1,
2, 4, 8, 16, 32, 64, 128, 256, 512, and 1024 contiguous page frames. These
blocks are used to find the best possible fit for the requested memory. For
example, 16 continuous page frames are requested. The algorithm goes to the
16-page-frame list and searches for free blocks in the list. If none are found,
the algorithm goes to the next list, the 32-page-frame list. This continues on
until a suitable block is found. From this, it can be determined that this
algorithm is an extension of the Best-Fit algorithm. Whenever a block is
allocated, the remaining free page frames are not left as internal fragments.
Rather, these contiguous sets of blocks are then placed back into the lists,
depending on the size left. If 32 page frames were requested and the only
contiguous set of blocks are found in the 128 list, the remaining 96 frames are
broken down into 64 continuous page frames and 32 continuous page frames. These
are then inserted into the corresponding list.

Once frames are released, the kernel tries to merge pairs of contiguous free
blocks of size b to form a single block, 2b. Once a pair, or a buddy, has been
found, the pair finds another pair that is of the same size as them and merges
together. The process is repeated until no pairs have been found and the largest
contiguous page frames are formed. \parencite[311-312]{linux}

Swapping is also a feature built into the Linux kernel. The swapping subsystem
does the following functions: Setup swap areas, manage the space on swap areas,
manage swapping in and out of RAM, and track swapped pages.
\parencite[713]{linux}

\subsection{File Management}
File management systems in both Fedora and openSUSE are, in essence, the same
since they both are built on the Linux kernel. Linux file management is built in
a single hierarchical directory structure. The structure begins with a root
directory, denoted by the forward slash /, which then expands into
sub-directories. All these other directories are descendants of root.
\parencite{linux_file_system_management}

File protection and directory protection in Linux systems are handled similarly
as well. Three fields --- owner, group, and universe --- are associated with
each file and directory. These fields consist of three bits “rwx” --- r for read
access, w for write access, and x for execution access. Each field separately
indicates the permissions of the file owner, the file’s group, and all the other
users using these three bits. Groups can only be created by the manager of the
facility or by any superuser. \parencite[553-554]{os}

Though both are similar, Fedora’s default and recommended system is Ext4 while
openSUSE Leap’s default file system is Btrfs. There are pros and cons to each
file system.

Most Linux distros use Ext4 as the default file system. It is a very robust file
system with impressive limits --- the largest volume/partition the Ext4 can make
is 1 exbibyte and the largest file size is 16 tebibytes --- but it is, however,
built on an aging code base. Ext4, like most file systems, also uses journaling
wherein changes to the file system are written sequentially to a journal.
Despite these features, it does not support transparent compression, transparent
encryption, data deduplication, or filesystem snapshots.
\parencite{ext4}

On the other hand, Btrfs --- which can be pronounced as “Butter FS” --- is a
newer system described as a Copy-on-Write (CoW) file system.This is a resource
management technique wherein when a process creates a child, it can exist as a
reference to the original data. Only once the child is modified is an actual
copy created and these modifications are only written on the copy
\parencite{cow}. The main difference between Btrfs and Ext4 is that Btrfs is
capable of addressing and managing more files, larger files, and larger volumes
than the Ext2, Ext3, and Ext4 file systems. It was made because developers
wanted to expand the functionalities of a modern file system to include
snapshots and checksums, among others. \parencite{ext4}

Snapshots are an important feature for a filesystem to have. Before doing any
risky modification, this feature allows you to take a snapshot of your
filesystem so that if an error occurs down the line, a snapshot of this early
state of your filesystem is available to restore. \parencite{shovon_1969}

Checksum is a block of data that is derived from another block of data to enable
error detection for a file or during a data transfer. The data is produced by
implementing a checksum algorithm to a block of data or a data packet. The
algorithm returns a value which is appended to the data packet (usually at the
end) and the receiver recalculates the checksums using the incoming data and
compares the two values. If there is a discrepancy between the values, this
means that an error occurred during data transmission. \parencite{isaac}

There is a clear advantage between Btrfs and Ext4. It was designed to have more
useful functionalities for a modern filesystem and it was built for
high-capacity and high-performance storage. However, there are still some pros
of Ext4 over the other. Despite being built on an aging code base, Ext4 is still
the most commonly used filesystem among Linux distros. This is because it has
proven to be stable, resilient, and reliable despite its simplicity. Due to
journaling support, data is kept safe even when there is an unexpected power
failure.

Though these are the defaults, both operating systems still support other file
systems such as Fedora still being capable of supporting Ext3, Ext2, and Btrfs.
This is also the same case for openSUSE with the addition of FAT, XFS, Swap, and
UDF. \parencite{fedora_file_system, suse_documentation,
  expert_partitioner}

\subsection{Resource Management}
Both Fedora and openSUSE share a way to manage resources called control groups
or cgroups for short which is a feature of the Linux kernel. Cgroups are a
collection of processes which can be bound to a set of parameters via the cgroup
system. It allows for allocating resources such as CPU time, network bandwidth,
and system memory among processes that are running on a system. Cgroups give
system administrators fine-tuned control over allocation, prioritization,
managing, and monitoring of system resources. This includes monitoring usage of
various types of resources, not allowing cgroups access to some resources, and
reconfiguring cgroups dynamically running on a system. \parencite{cgroups,
  cgroups_linux_manual_page}

Cgroups are organized hierarchically and child cgroups inherit some of the
attributes of their parent cgroup. The cgroup model is “one or more separate,
unconnected trees of tasks or processes”. It is different to a single tree of
processes such that there can be many different cgroups hierarchies
simultaneously on a given system.\parencite{cgroups}

\subsection{Concurrency Management}
Concurrency management for Fedora and openSUSE is also essentially the same
since they share the same kernel. Multitasking in operating systems lets
multiple threads execute in parallel to optimize system performance. However, if
this is not implemented cautiously, it could lead to issues such as processes
sharing the same resources at the same time. This is called the critical section
of the process. Processes should be able to use the same resource but only one
process at a time. In order to address these concurrency issues that could arise
from multitasking, Linux kernel uses two synchronization mechanisms called Mutex
(Mutual Exclusion Object) and Semaphore. \parencite{concurrency,
  concurrency_linux}

Mutex is used to allow only one process at a time to have access to a specific
resource. This allows all processes to use this resource but only one at a time.
In order to handle the critical section issue, Mutex uses the lock-based
technique. When a process requests to use a specific resource, a mutex object
will then be generated by the system with a unique ID.\@ This then allows the
process to occupy a lock on the object whenever it wishes to use the resource.
Once the process is finished it releases the mutex object and other processes
can then create the mutex object and use it in the same manner. Locking the
object allocates that resource to the specific process and other processes are
essentially “locked out” and cannot use that resource.

Semaphore is used for process synchronization wherein an integer variable S is
initialized with the number of resources present in the system. Two functions,
such as \texttt{wait()} and \texttt{signal()}, are then used to modify the value
of this variable but only one process at a time is allowed to change this value.
The two categories of semaphores are counting semaphores and binary semaphores.

Counting semaphores have the semaphore variable initialized with the number of
resources available. This is so that whenever a process takes a resource to use
it, the value of the semaphore variable is decreased by one by the
\texttt{wait()} function. After the resource is released, the \texttt{signal()}
function then increases the value of the semaphore by 1 again. When the
semaphore variable value reaches 0 this means that all resources are used and
there is none left to be used. A process that wishes to use a resource has to
wait for its turn.

Binary semaphores are implemented through setting the semaphore value to 0 or 1.
It is first initialized to 1 and if a process requests some resource, the
variable value is then changed to 0. When the resource is released, the value is
then increased to 1. This is similar to Mutex wherein a process cannot use a
certain resource at a particular instant of time when the value of the semaphore
is 0 and has to wait for the process using the resource to release it. No
locking is performed however. \parencite{concurrency}

\section{Conclusion}
The main difference between the two operating systems lies in userland
applications, and supported architecture. The underlying kernel that runs the
major components of the operating system of the two systems is the same, the
Linux kernel. However, there are minor differences between the two in the way
they configure the kernel because Fedora does not support 32-bit processors
while openSUSE does. This results in less addressable space and supported RAM
capacity for 32-bit openSUSE. In addition, the default file systems for both
operating systems are different. Fedora recommends ext4, while openSUSE defaults
to btrfs. Other than those differences, the rest of the kernel remains the same
for the two.

Because of these, choosing one operating system over the other depends on the
following considerations: hardware, use case, preference of userland
applications, and release styles. If the user’s hardware is 32-bit, then they
have no choice but to use openSUSE, otherwise if they have 64-bit processors,
both operating systems are available. If the user’s main use case is for cloud,
or IoT, Fedora has an edge with their specific editions. For general server and
desktop use, both operating systems would be good choices. Both operating
systems have more or less the same applications, however the package manager
between the two differs. Fedora uses dnf, while openSUSE uses YaST and ZYpper.
Finally, the stability of the operating system differs between the two. openSUSE
uses a rolling release model, while Fedora releases biannually. If stability is
more important, Fedora takes the lead. If getting the latest version of software
is more important, openSUSE takes the edge.

All in all, there really is not a lot of difference between Fedora and openSUSE
when the way they manage the different resources of a computer is compared.
Choosing one operating system over the other mostly comes down to user
preference and available hardware.

\newpage
\nocite{*} \printbibliography[heading=bibintoc,title={References}]{}

\newpage
\section{Appendix}
\subsection{Source Code}
This project was written in \LaTeX{}. The source code of this document can be
found at GitHub: \url{https://github.com/ohhskar-school/cmsc-125-final-project}.
The link for Overleaf can be found here:
\url{https://www.overleaf.com/project/60b74ef310d0ee358da40652}
\subsection{Division of Task}
\subsubsection{Tumulak, Patricia Lexa U.}
\begin{itemize}
  \item{openSUSE}
  \item{File Management}
  \item{Resource Management}
  \item{Concurrency Management}
\end{itemize}
\subsubsection{Valles, Oscar Vian L.}
\begin{itemize}
  \item{Fedora}
  \item{Kernel}
  \item{Processor Management}
  \item{Memory Management}
  \item{Conclusion}
\end{itemize}

\end{document}
